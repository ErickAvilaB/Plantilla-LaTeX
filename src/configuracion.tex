% Configuración del documento

% ---------- Paquetes necesarios ----------
\usepackage[utf8]{inputenc} % Soporte para caracteres
\usepackage[spanish,mexico]{babel} % Soporte en español
\usepackage[letterpaper, margin=2cm]{geometry} % Soporte para tamaño de hoja y margenes
\usepackage{graphicx} % Soporte para imágenes
\usepackage{amsmath} % Soporte para matemáticas
\usepackage{amssymb} % Soporte para símbolos
\usepackage{amsthm} % Soporte para teoremas
\usepackage{setspace} % Soporte para entrelineado
\usepackage{fancyhdr} % Soporte para encabezados y pie de pagina
\usepackage{lipsum} % Texto de relleno
\usepackage{xcolor} % Para definir colores
\usepackage[normalem]{ulem} % Para subrayado sin cambiar énfasis
\usepackage[colorlinks=true, linkcolor=black, urlcolor=blue, citecolor=blue, filecolor=blue]{hyperref}
\usepackage{lastpage}

% ---------- Configuraciones ----------
\graphicspath{ {img/} } % Carpeta de imágenes
\onehalfspacing{} % Entrelineado 1.5

% Estilo de la pagina
\fancypagestyle{miEstilo}{
    \fancyhf{} % Resetea estilos
    \fancyhead[L]{\autorAPA} % Nombre del autor
    \fancyhead[R]{\tarea} % Nombre de la tarea
    \renewcommand{\headrulewidth}{0.01cm} % Linea arriba
    \renewcommand{\footrulewidth}{0.01cm} % Linea abajo
    \fancyfoot[C]{\textit{Página \thepage \hspace{1pt} de \pageref{LastPage}}} % Número de página
}

\fancypagestyle{portada}{
    \fancyhf{} % Resetea estilos
    \renewcommand{\headrulewidth}{0.01cm} % Linea arriba
    \renewcommand{\footrulewidth}{0.01cm} % Linea abajo
    \fancyfoot[C]{\textit{Página \thepage \hspace{1pt} de \pageref{LastPage}}} % Número de página
}

\thispagestyle{portada} % Estilo primera pagina
\pagestyle{miEstilo} % Establece el estilo de la pagina

% Hace que las demostraciones sean en cursiva, negritas y subrayadas
\def\proof{\textit{\textbf{\underline{Demostración}}}\\[0.2cm]\indent}
\def\endproof{\hspace*{\fill} $_\blacksquare$}

% ---------- Comandos ----------

% Comando personalizado para subrayar enlaces
\newcommand{\ulhref}[2]{\href{#1}{\uline{#2}}}

% Comando personalizado para generar el cuerpo basico del documento
\newcommand{\cuerpo}{
    % ---------- Cuerpo del documento ----------

    \section*{Introducción}

    \lipsum[1]
    \vspace{0.5cm}

    \begin{proof}
        \lipsum[2]
    \end{proof}
    \vspace{0.5cm}

    Checa este video: \ulhref{https://www.youtube.com/watch?v=r4l9bFqgMaQ}{Frank
        Ocean \- Nights} }